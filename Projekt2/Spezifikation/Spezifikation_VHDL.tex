
\documentclass[12pt,a4paper]{article}

\usepackage[german]{babel}
\usepackage[utf8x]{inputenc}
\usepackage{amsmath}
\usepackage{enumitem}
\usepackage{graphicx}
\usepackage{fancyhdr}
\usepackage{longtable}
\usepackage{lastpage}


\title{Spezifikation VHDL}
\author{Amar Saljic, Arseniy Vershinin, Jonathan Kienzle}
\date{08. Juni 2013}

\pagestyle{fancy}
\fancyhead{} % clear all header fields
\fancyhead[RE,LO]{\emph{Spezifikation VHDL - Gruppe 50}}
\renewcommand{\headrulewidth}{0pt} % no line in header area
\fancyfoot{} % clear all footer fields
\fancyfoot[LE,RO]{\thepage /\pageref{LastPage}}           % page number in "outer" position of footer line
%\fancyfoot[RE,LO]{\emph{Spezifikation VHDL - Gruppe 50}} % other info in "inner" position of footer line

\begin{document}
\maketitle

\thispagestyle{fancy}

\section{Architektur}

Der Timing-Generator wird als Zustandsautomat implemetiert, der anhand des 5.6MHz Taktsignals, das auf dem Eingang \emph{CLK} liegt, seinen Zustand wechselt. Dabei werden in jedem Zustand alle Ausgänge passend gesetzt. Das \emph{RESET}-Signal sorgt dafür, dass alle Signale auf den Wert 0 gesetzt werden. Wir gehen davon aus, dass der SPDIF-Konverter, der vom Timing-Generator gesteuert wird, nur mit 16-bitigen Samples arbeitet.

\section{Verhalten}

\subsection{Lösungsvorschlag 1}

Für die Entscheidung über einen Zustandsübergang werden drei Zähler angelegt. Sobald ein Zähler das Ende seines Wertebereiches erreicht, wird er mit der nächsten steigenden Flanke des \emph{CLK}-Signals wieder auf 0 gesetzt.

\begin{itemize}
\item {\bf SUBFRAME\_CLK\_C}: Dieser Zähler wird mit jeder steigenden Flanke des CLK-Signals inkrementiert und dient der Unterscheidung zwischen den Arten des zu generierenden Bits eines Subframes (Preambel-Bit, Audio Data Bit, V-Bit, U-Bit, P-Bit). Da die Größe eines Subframes 32 Bits beträgt und der Generator mit doppelter Frequenz getaktet ist, wird ein Subframe über 64 Takte generiert.\newline Wertebereich: 0-63.
\item {\bf SUBFRAMES\_C}: Die Nummer eines Subframes innerhalb eines Frames. Wird inkrementiert sobald das Signal \emph{SUBFRAME\_CLK\_C} seinen Wert von 63 auf 0 wechselt.\newline Wertebereich: 0-1.
\item {\bf FRAMES\_C}: Die Anzahl von Frames innerhalb eines Blocks. Wird inkrementiert sobald das Signal \emph{SUBFRAMES\_C} seinen Wert auf 0 wechselt (ein ganzes Frame wurde abgeschlossen). \newline Wertebereich: 0-191. 
\end{itemize}

\noindent
Folgende Tabelle veranschaulicht die Steuerung der vorhandenen Ausgänge, deren Werte hauptsächlich vom Zustand des \emph{SUBFRAME\_CLK\_C} Zählers abhängen. Andere für die Schaltung entscheidende Zählerwerte werden in der Spalte "`Prozess"' genannt. Alle nicht beteiligten und nicht explizit auf 1 gesetzten Ausgänge werden auf den Wert 0 gesetzt.

\begin{center}
  \begin{longtable}{| c | p{2cm} | p{2.1cm} | p{5cm} |}
    \hline
	SUBFRAME\_CLK\_C & Subframe Teil & Beteiligte Ausgänge & Prozess \\
    \hline
	0..7 & Preambel & X,Y,Z & 
    Wenn \emph{FRAMES\_C} = 0, dann wird Z auf 1 gesetzt. Anderenfalls gilt:\newline Wenn \emph{SUBFRAMES\_C} = 0 (erstes Subframe, linker Kanal), dann wird X auf 1 gesetzt.\newline Wenn \emph{SUBFRAMES\_C} = 1 (zweites Subframe, rechter Kanal), dann wird Y auf 1 gesetzt. \\
    \hline
	8..23 & Audio Data (Verschiebung) & SHIFTCLK & 
    \emph{SHIFTCLK} wird über 16 Takte auf 1 gesetzt, um dem Data-Multiplexer zu signalisieren, dass das 16-bitige Sample um 8 Bit verschoben werden muss. \\
    \hline
	24..55 & Audio \newline Data & LOAD\_L LOAD\_R & 
    Wenn \emph{SUBFRAMES\_C} = 0 (linker Kanal wird bearbeitet), dann wird \emph{LOAD\_L} auf 1 gesetz.\newline Anderenfalls (rechter Kanal wird bearbeitet) wird \emph{LOAD\_R} auf 1 gesetzt.  \\
    \hline
	56..61 & Zusätzliche Bits & START & 
    In dieser Phase wird dem Subcode-Generator mit \emph{START} = 1 signalisiert, dass die User Data, Channel Status und Validity Bits generiert werden sollen.\\
    \hline
	62..63 & Parity-Bit & P &
    Zum Abschluss soll der Biphase-Mark-Encoder das Parity-Bit generieren, was mit \emph{P} = 1 signalisiert wird. \\
    \hline
  \end{longtable}
\end{center}

\subsection{Lösungsvorschlag 2}

Statt drei verschiedene Zähler zu verwalten, kann ein allgemeiner Zähler \emph{CLK\_C} verwendet werden, der jeden Takt inkrementiert wird. Die Werte der separaten Zähler können dann wie folgt berechnet werden:

\begin{itemize}
\item \emph{SUBFRAME\_CLK} = \emph{CLK\_C} mod 64.
\item \emph{SUBFRAMES\_C} = (\emph{CLK\_C} mod 128) \(>\) 63.
\item \emph{FRAMES\_C} = (\emph{CLK\_C} div 128) mod 192 (Blockgröße beträgt 192)
\end{itemize}

Wertebereich von \emph{CLK\_C}: 0-24575 \((2 * 64 * 192)\).


\subsection{Bewertung der Ansätze}

\begin{center}
  \begin{tabular}{| l | p{6cm} | p{6cm} |}
    \hline
	 & Ansatz 1 & Ansatz 2 \\
    \hline
	 Vorteile &  
     \begin{itemize}[leftmargin=0.4cm]
		\item Geringerer Speicherbedarf: Aus den Wertebereichen der Zähler folgt, dass die notwendige Speichergröße 15 Bit beträgt.
		\item Höhere Effizienz: Inkrement-Operation sehr günstig.
	 \end{itemize}
     &  
     \begin{itemize}[leftmargin=0.4cm]
		\item Einfachere Implementierung: Es müssen keine separaten Zähler mit Obergrenzenüberprüfung gehandhabt werden, da die mod-Operation sicherstellt, dass die entsprechenden Grenzen nicht überschritten werden.
	 \end{itemize}
     \\
    \hline
    
	 Nachteile &
     \begin{itemize}[leftmargin=0.4cm]
		\item Beim Inkrementieren der Zähler muss jedes Mal ein Vergleich mit der Obergrenze durchgeführt werden.
	 \end{itemize}
     &
     \begin{itemize}[leftmargin=0.4cm]
        \item Geringere Effizienz: Die Operationen mod und div sind vergleichsweise sehr teuer.
        \item Erhöhter Speicherbedarf: Der Zähler \emph{CLK\_C} benötigt zusätzliche 15 Bit Speicher.
	 \end{itemize}
     \\
     \hline
  \end{tabular}
\end{center}

\noindent
Es wird Lösungsvorschlag 1 implementiert, da die Vorteile eindeutig überwiegen.

\end{document}
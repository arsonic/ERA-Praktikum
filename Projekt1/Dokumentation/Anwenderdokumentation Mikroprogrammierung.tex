%
% Hello! Here's how this works:
%
% You edit the source code here on the left, and the preview on the
% right shows you the result within a few seconds.
%
% Bookmark this page and share the URL with your co-authors. They can
% edit at the same time!
%
% You can upload figures, bibliographies, custom classes and
% styles using the files menu.
%
% If you're new to LaTeX, the wikibook at
% http://en.wikibooks.org/wiki/LaTeX
% is a great place to start, and there are some examples in this
% document, too.
%
% Enjoy!
%
\documentclass[12pt,a4paper]{article}

\usepackage[german]{babel}
\usepackage[utf8x]{inputenc}
\usepackage{amsmath}
\usepackage{enumitem}
\usepackage{graphicx}
\usepackage{fancyhdr}
\usepackage{longtable}
\usepackage{lastpage}
\usepackage[bottom]{footmisc}
\usepackage{listings}
\usepackage{rotating}
\usepackage{changepage}   % for the adjustwidth environment
\usepackage[parfill]{parskip}
\usepackage[table]{xcolor}

\title{Anwenderdokumentation Mikroprogrammierung}
\author{Amar Saljic, Arseniy Vershinin, Jonathan Kienzle}
\date{12. Juli 2013}

\pagestyle{fancy}
\fancyhead{} % clear all header fields
\fancyhead[RE,LO]{\emph{Anwenderdokumentation Mikroprogrammierung - Gruppe 50}}
\renewcommand{\headrulewidth}{0pt} % no line in header area
\fancyfoot{} % clear all footer fields
\fancyfoot[LE,RO]{\thepage /\pageref{LastPage}}           % page number in "outer" position of footer line
%\fancyfoot[RE,LO]{\emph{Anwenderdokumentation Mikroprogrammierung - Gruppe 50}} % other info in "inner" position of footer line

\begin{document}
\definecolor{lightgray}{gray}{0.9}
\maketitle

\thispagestyle{fancy}

\section{Übersicht}

Das Programm \emph{strlen}, welches die Länge einer Zeichenkette in Zeichen ermittelt, wurde auf zwei Arten implementiert:
\begin{enumerate}
\item Als Maschinenprogramm
\item Als spezieller Maschinenbefehl in einem gesonderten Mikroprogramm
\end{enumerate}
Die zu zählende Zeichenkette liegt im Unicode-Format (16 bit pro Zeichen) im Hauptspeicher und das abschließende Nullzeichen (0x0000) wird nicht mitgezählt.

\section{Black-Box Beschreibung}

\subsection{Maschinenprogramm}
 	
Das Maschinenprogramm belegt im Hauptspeicher den Adressbereich 0x0000 - 0x0009. Um die Länge einer Unicode-Zeichenkette zu ermitteln, muss das Programm folgendermaßen eingesetzt werden:

\emph{Vorbedingungen:}
\begin {enumerate}
\item Maschinenbefehlszähler ist auf 0 gesetzt.
\item Register \emph{r0} enthält die Startadresse der Zeichenkette.
\end{enumerate}

\emph{Ausgabe:} Register \emph{r0} enthält die Länge der Zeichenkette.

\subsection{Mikroprogramm}
Das Mikroprogramm belegt im Mikroprogrammspeicher den Adressbereich 0x100 - 0x105. Die Startadresse entspricht also dem Opcode 10. Um die Länge einer Unicode-Zeichenkette zu ermitteln, muss das Programm folgendermaßen eingesetzt werden:

\emph{Vorbedingungen:}
\begin {enumerate}
\item Register \emph{ra} enthält die Startadresse der Zeichenkette.
\item Die Nummer des Registers \emph{ra} ist dem Mikroprogramm als Quellopearand (RA) zu übergeben. Der Wert des Zieloperanden (RB) hat keinen Einfluss auf das Ergebnis. 
\end{enumerate}

\emph{Ausgabe:} Register \emph{r1} enthält die Länge der Zeichenkette.

\vspace{0.3cm}
\emph{Anmerkung}:
\begin {itemize}
\item Register \emph{r1} darf nicht als Quellopearand (RA) übergeben werden, da es als Ergebnis-Register eingesetzt wird.
\end{itemize}

\section{Beispielszenarien}

%--------------------------------------------------------
\subsection{Maschinenprogramm}
\emph{Ziel}: Bestimmung der Länge der Zeichenkette 'Affe', welche ab Adresse 0x000B im Hauptspeicher liegt.

\emph{Schritte}:

\begin {enumerate}
\item Maschinenbefehlszähler auf 0 setzen.
\item Register \emph{r0} auf den Wert 0x000B (Startadresse der Zeichenkette) setzen.
\item Das Programm solange laufen lassen, bis der Maschinenbefehlszähler den Wert 0x000A erreicht (Ende des Programms).
\end{enumerate}

\emph{Ergebnis:} \emph{r1} enthält den Wert 4.

%--------------------------------------------------------
\subsection{Mikroprogramm}

\emph{Ziel}: Bestimmung der Länge der Zeichenkette 'Elefant', welche ab Adresse 0x000D im Hauptspeicher liegt. Die Startadresse der Zeichenkette liegt in Register \emph{r2}. Der Mikroprogrammaufruf soll in der Hauptspeicherzelle 0x000B stehen.

\emph{Schritte}:

\begin {enumerate}
\item Maschinenbefehlszähler auf 0x000B setzen.
\item Dem Mikroprogramm \emph{r2} als Quelloperanden (RA) übergeben: Inhalt der Speicherzelle 0x000B auf 0x1020 setzen.
\item IFETCH ausführen.
\end{enumerate}

\emph{Ergebnis:} \emph{r1} enthält den Wert 7.

\end{document}
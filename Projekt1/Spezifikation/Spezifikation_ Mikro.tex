%
% Hello! Here's how this works:
%
% You edit the source code here on the left, and the preview on the
% right shows you the result within a few seconds.
%
% Bookmark this page and share the URL with your co-authors. They can
% edit at the same time!
%
% You can upload figures, bibliographies, custom classes and
% styles using the files menu.
%
% If you're new to LaTeX, the wikibook at
% http://en.wikibooks.org/wiki/LaTeX
% is a great place to start, and there are some examples in this
% document, too.
%
% Enjoy!
%
\documentclass[12pt,a4paper]{article}

\usepackage[german]{babel}
\usepackage[utf8x]{inputenc}
\usepackage{amsmath}
\usepackage{enumitem}
\usepackage{graphicx}
\usepackage{fancyhdr}
\usepackage{longtable}
\usepackage{lastpage}
\usepackage[bottom]{footmisc}


\title{Spezifikation Mikroprogrammierung}
\author{Amar Saljic, Arseniy Vershinin, Jonathan Kienzle}
\date{08. Juni 2013}

\pagestyle{fancy}
\fancyhead{} % clear all header fields
\fancyhead[RE,LO]{\emph{Spezifikation Mikroprogrammierung - Gruppe 50}}
\renewcommand{\headrulewidth}{0pt} % no line in header area
\fancyfoot{} % clear all footer fields
\fancyfoot[LE,RO]{\thepage /\pageref{LastPage}}           % page number in "outer" position of footer line
%\fancyfoot[RE,LO]{\emph{Spezifikation Mikroprogrammierung - Gruppe 50}} % other info in "inner" position of footer line

\begin{document}
\maketitle

\thispagestyle{fancy}

\section*{Allgemeines}

Das Programm \textit{strlen} wird auf folgende zwei Arten implementiert: \newline
1. als Maschinenprogramm\newline
2. als spezieller Maschinenbefehl in einem gesonderten Mikroprogramm \newline 

\section{Maschinenprogramm}

\subsection{Architektur}
Die Funktionalität von \textit{strlen} wird als Maschinenprogramm implementiert, welches die folgenden Maschinenbefehle aufruft: 

\begin{itemize}
\item move
\item cmp
\item jmpcc
\item inc
\item sub
\item halt
\end{itemize}
Diese werden von der Gruppe, so wie im Dokument \emph{``miziel.ps''} spezifiziert, als Mikroprogramme implementiert. Vor Aufruf des Programms sollte die Startadresse des Zeichenfeldes in \textit{r0} abgelegt werden. Nach Ablauf des Programms steht die Länge der Zeichenkette in \textit{r1}. 

\subsection{Verhalten}
Der Aufbau des Maschinenprogramms sieht wie folgt aus: 
\begin {enumerate}
\item \textit{r1} (Zähler) auf 0 setzen
\item \textit{r0} (Startadresse) nach \textit{r3} (Zeichenadresse) kopieren
\item Vergleiche \lbrack \textit{r3}\rbrack \footnotemark[1] \ mit dem Nullzeichen \((0x0000)\), welches das Ende einer Zeichenkette markiert
\begin {enumerate}
\item \(\lbrack \textit {r3}\rbrack \ == 0x0000 \rightarrow\) beende Maschinenprogramm
\item \(\lbrack \textit{r3}\rbrack \hspace{6 pt} \ != 0x0000 \rightarrow\) \textit{r1} (Zähler) um 1 inkrementieren
\end {enumerate}
\item \textit{r3} (Zeichenadresse) inkrementieren. Da die Größe einer Hauptspeicherzelle der Zielmaschine genau der Größe eines Unicode-Zeichens entspricht, muss \textit{r3} um 1 inkrementiert werden. 
\item Springe zu Schritt 3
\end {enumerate}

\subsection{Alternativer Lösungsansatz}
Die Länge der Zeichenkette lässt sich über die Differenz von Start -und End-adresse berechnen. Dabei adressiert die Endadresse genau die Zelle, die das Nullzeichen \((0x0000)\) enthält.
\begin{enumerate}
\item \textit{r0} (Startadresse des Zeichenfeldes) nach \textit{r1} kopieren
\item Vergleiche \lbrack \textit{r1}\rbrack \ mit dem Nullzeichen \((0x0000)\)
\begin{enumerate}
\item \(\lbrack \textit{r1}\rbrack \ == 0x0000 \rightarrow \) springe zu Schritt 4
\item \(\lbrack \textit{r1}\rbrack \hspace{6 pt} \ != 0x0000 \rightarrow \textit{r1} \) (Zeichenadresse) um 1 inkrementieren
\end{enumerate}
\item Springe zu Schritt 2
\item Speichere \textit{r1} (Endadresse) \(-\) \textit{r0} (Startadresse) in \textit{r1}
\item Beende Maschinenprogramm
\end {enumerate}
\footnotetext[1]{[\textit{ra}] - Inhalt der Speicherzelle, deren Adresse in Register \textit{ra} abgelegt ist.} 
\subsection{Analyse der Ansätze}
Alternative 2 benötigt ein Register weniger, da kein Zähler verwendet wird. Außerdem spart sie in jedem Schleifendurchlauf 4 Takte (IFETCH + inc-Operation), welche in Alternative 1 für die Inkrementierung des Zählers benötigt werden.\newline \newline
Aufgrund der erhöhten Effizienz wird Alternative 2 implementiert.
\newpage
\section{Mikroprogramm}

\subsection{Architektur}
Das Programm \textit{strlen} wird als eigenständiges Mikroprogramm implementiert, welches die Länge einer gegebenen Zeichenkette berechnet und im Register \textit{r1} speichert. Dabei muss die Startadresse des Zeichenfeldes, wie in den zwei Lösungsansätzen spezifiziert, übergeben werden.


\subsection{Verhalten}
Der Aufbau des Mikroprogramms sieht wie folgt aus: 
\begin {enumerate}
\item Startadresse des Zeichenfeldes, welche \emph{als Immidiate} übergeben wird, nach \textit{r1} kopieren (Befehlszähler erhöhen und Hauptspeicher adressieren)
\item \textit{r1} nach \textit{r8} kopieren, um die Startadresse zu sichern
\item Vergleiche \lbrack \textit{r1}\rbrack \ mit dem Nullzeichen \((0x0000)\)
\begin {enumerate}
\item \(\lbrack \textit{r1}\rbrack \ == 0x0000 \rightarrow \) Springe zu Schritt 5
\item \(\lbrack \textit{r1}\rbrack \hspace{6 pt} \ != 0x0000 \rightarrow \) \textit{r1} (Zeichenadresse) um 1 inkrementieren
\end {enumerate}
\item Springe zu Schritt 3
\item Speichere \textit{r1} (Endadresse) \(-\) \textit{r8} (Startadresse) in \textit{r1}
\item Springe zu IFETCH
\end {enumerate}

\subsection{Alternativer Lösungsansatz}
Das Argument (Startadresse des Zeichenfeldes) lässt sich alternativ \emph{über ein Register}, welches im Folgenden mit \textit{ra} bezeichnet wird, übergeben. 
\begin{enumerate}
\item \textit{ra} (Startadresse des Zeichenfeldes) nach \textit{r1} kopieren
\item Vergleiche \lbrack \textit{r1}\rbrack \ mit dem Nullzeichen (0x0000)
\begin{enumerate}
\item \(\lbrack \textit{r1}\rbrack == 0x0000 \rightarrow \) springe zu IFETCH
\item \(\lbrack \textit{r1}\rbrack \hspace{6 pt} != 0x0000 \rightarrow \) \textit{r1} (Zeichenadresse) inkrementieren
\end{enumerate}
\item Springe zu Schritt 2
\item Speichere \textit{r1} (Endadresse) \(-\) \textit{ra} (Startadresse) in \textit{r1}
\item Springe zu IFETCH
\end {enumerate}
\subsection{Analyse der Ansätze}
Alternative 2 bietet dem Maschinenprogrammierer mehr Flexibilität, da er die Startadresse des Zeichenfeldes in einem Register übergeben kann. Mit Alternative 1 müsste er die Startadresse schon vor dem Ausführen des Programms kennen, was häufig nicht der Fall ist.\newline \newline
Aus den oben genannten Gründen wird Alternative 2 implementiert. 

\end{document}